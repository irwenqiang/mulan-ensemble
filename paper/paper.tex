\documentclass{acm_proc_article-sp}

% scientific notation, 1\e{9} will print as 1x10^9
\providecommand{\e}[1]{\ensuremath{\times 10^{#1}}}
\usepackage{amsmath} % needed for pmatrix
\usepackage{booktabs} % Fancy tables

\begin{document}

\title{Multi-Label Text Categorization via Ensemble Techniques}

\numberofauthors{3}
\author{
% 1st. author
\alignauthor
Martin Boro\v{s}\
\
       \affaddr{Some}\\
       \affaddr{Martin's}\\
       \affaddr{address}\\
       \email{borosko@gmail.com}
% 2nd. author
\alignauthor
Franky\\
       \affaddr{Some}\\
       \affaddr{Franky's}\\
       \affaddr{address}\\
       \email{franky.id@gmail.com}
% 3rd. author
\alignauthor
Ji\v{r}\'{i} Mar\v{s}\'{i}k\\
       \affaddr{Some}\\
       \affaddr{Ji\v{r}\'{i}'s}\\
       \affaddr{address}\\
       \email{jiri.marsik89@gmail.com}}

\date{30 November 2011}

\maketitle
\begin{abstract}
The place for our abstract. Estimated 1/4 of a page.
\end{abstract}

\category{H.3}{Informationg Storage and Retrieval}{Content Analysis and Indexing}

\terms{Experimentation, Performance}

\keywords{text categorization, multi-label classification, ensemble techniques}

\section{Introduction}
What did we set out to do, i.e. what problem are we solving (can talk about
dataset or about the text categorization in general) and why are we trying to do
what we are trying to do (ensemble techniques look good). Estimated 1/2 of a
page, possibly more.

\section{Previous Work}
Some summary of results of ensemble techniques and possibly papers of other
people working on the same dataset as us. Estimated 1/2 of a page.

\section{Multi-label Classification Algorithms}
Some short mention about multi-label classification algorithms, mostly based on
the slides about multi-label classification. This should very shortly introduce
the individual multi-label classifiers we will be using. Estimated 1/2 of a page.

\subsection*{Ensemble Techniques}
Introduce the ensemble techniques in a similar fashion as in the original
article (citing the original paper). Estimated 1/2 of a page.

\section{Experiment Setup}
This would be the place to mention the implementation of the ensemble
techniques, the use of Mulan, the selection and format of the data. Estimated
1/4 of a page.

\section{Results}
Here will be the tables containing the results of our experiments and some
analysis of the results. Estimated 1 page.

\begin{table*}[ht]
	\begin{center}
	\label{tab:bipartition-table}
		\begin{tabular}{rccccccc}
		\toprule
			 & HL & SA & Recall & Accu. & MicroP & MicroR & MicroF_1\\
			\midrule
			RAkEL & $0.0702$ & $0.2310$ & $0.5976$ & $0.4967$ & $0.6811$ & $0.5578$ & $0.6127$\\
			CLR & $0.0699$ & $0.2349$ & $0.6841$ & $\textbf{0.5322}$ & $0.6503$ & $0.6475$ & $\textbf{0.6485}$\\
			ML-kNN & $0.0762$ & $0.1697$ & $0.5041$ & $0.4335$ & $0.6666$ & $0.4720$ & $0.5518$\\
			HOMER & $0.0799$ & $0.2115$ & $0.6604$ & $0.5029$ & $0.5958$ & $0.6218$ & $0.6081$\\
			IBLR & $0.0752$ & $0.1772$ & $0.5209$ & $0.4429$ & $0.6679$ & $0.4890$ & $0.5642$\\
			Intersection & $0.0772$ & $0.1433$ & $0.3173$ & $0.3078$ & $\textbf{0.8354}$ & $0.2823$ & $0.4214$\\
			Union & $0.0929$ & $0.1513$ & $\textbf{0.8474}$ & $0.5202$ & $0.5221$ & $\textbf{0.8194}$ & $0.6374$\\
			Majority vote & $\textbf{0.0639}$ & $\textbf{0.2559}$ & $0.6153$ & $0.5300$ & $0.7300$ & $0.5709$ & $0.6402$\\
		\bottomrule
		\end{tabular}
                \caption{Experiment results for the bipartition-based ensemble techniques.}
	\end{center}
\end{table*}

\begin{table*}[ht]
	\begin{center}
	\label{tab:score-table}
		\begin{tabular}{rccccccc}
		\toprule
			 & AP & CO & OE & IE & ESS & RL & MicroAUC\\
			\midrule
			RAkEL & $0.7385$ & $5.9043$ & $0.2469$ & $0.5898$ & $5.9793$ & $0.1263$ & $0.8673$\\
			CLR & $0.8023$ & $2.8701$ & $0.2349$ & $\textbf{0.5067}$ & $2.2828$ & $0.0507$ & $0.9399$\\
			ML-kNN & $0.7204$ & $4.2654$ & $0.3126$ & $0.6371$ & $4.0329$ & $0.0930$ & $0.9048$\\
			HOMER & $0.6276$ & $8.8283$ & $0.3579$ & $0.6805$ & $10.7272$ & $0.2261$ & $0.7875$\\
			IBLR & $0.7317$ & $4.0045$ & $0.2972$ & $0.6222$ & $3.7347$ & $0.0855$ & $0.9098$\\
			Minimum & $0.6344$ & $9.3977$ & $0.2747$ & $0.6765$ & $11.5653$ & $0.2450$ & $0.7716$\\
			Maximum & $0.7535$ & $2.9771$ & $0.3484$ & $0.5734$ & $2.6181$ & $0.0585$ & $0.9328$\\
			Mean & $0.8039$ & $\textbf{2.8153}$ & $0.2314$ & $0.5102$ & $2.2100$ & $0.0496$ & $\textbf{0.9477}$\\
			Top\textsubscript{3} & $\textbf{0.8082}$ & $2.8228$ & $\textbf{0.2140}$ & $0.5082$ & $\textbf{2.2085}$ & $\textbf{0.0495}$ & $0.9475$\\
		\bottomrule
		\end{tabular}
                \caption{Experiment results for the score-based ensemble techniques.}
	\end{center}
\end{table*}


\section{Conclusion}
A summary of what we observed in the results and the conclusions reached from
there. Estimated 1/2 of a page (including the references).

% The following two commands are all you need in the
% initial runs of your .tex file to
% produce the bibliography for the citations in your paper.
\bibliographystyle{abbrv}
\bibliography{ensemble}  % ensemle.bib is the name of the Bibliography in this case
% You must have a proper ".bib" file
%  and remember to run:
% latex bibtex latex latex
% to resolve all references

\balancecolumns
% That's all folks!
\end{document}
