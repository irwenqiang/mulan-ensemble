\documentclass{acm_proc_article-sp}

% scientific notation, 1\e{9} will print as 1x10^9
\providecommand{\e}[1]{\ensuremath{\times 10^{#1}}}
\usepackage{amsmath} % needed for pmatrix
\usepackage{booktabs} % Fancy tables
\usepackage{fixltx2e}

\begin{document}

\title{Multi-Label Text Classification via Ensemble Techniques}

\numberofauthors{3}
\author{
% 1st. author
\alignauthor
Martin Boro\v{s}\\
       \affaddr{Shanghai Jiao Tong University}\\
       \affaddr{800 Dongchuan Road, Shanghai, 200240}\\
       \email{borosko@gmail.com}
% 2nd. author
\alignauthor
Franky\\
       \affaddr{Shanghai Jiao Tong University}\\
       \affaddr{800 Dongchuan Road, Shanghai, 200240}\\
       \email{franky.id@gmail.com}
% 3rd. author
\alignauthor
Ji\v{r}\'{i} Mar\v{s}\'{i}k\\
       \affaddr{Shanghai Jiao Tong University}\\
       \affaddr{800 Dongchuan Road, Shanghai, 200240}\\
       \email{jiri.marsik89@gmail.com}}

\date{30 November 2011}

\maketitle
\begin{abstract}
Text classification is one of the important problems being solved in
information retrieval. However, traditional single-label classifiers
are no longer sufficient and multi-label approaches are becoming more
relevant. There have been a lot of proposals for multi-label learners
and in our work, we tried applying ensemble techniques, which have
proven to be effective in solving other multi-label classification
problems, to combine them. We implemented seven ensemble techniques
presented in previous work and evaluated their performance. We have
found that some of the ensemble classifiers outperform all of the
individual classifiers. Ensemble techniques have thus proven
themselves to be applicable to the domain of text classification.
\end{abstract}

\category{H.3}{Information Storage and Retrieval}{Content Analysis and Indexing}

\terms{Experimentation, Performance}

\keywords{text classification, multi-label classification, ensemble techniques}

\section{Introduction}
In many fields, the labeled data may be insufficient in quantity while the unlabeled may be vast. When it comes to the domain of textual data, the reasons for classification of newspaper articles, academic papers or aviation safety reports as in our work are obvious. Manual classification of existing documents or of the extensive amount of newly created documents is unfeasible. It is not only time consuming and expensive, manual annotators may also produce diverse and inadequate classifications. All these limitations have led to the development of automatic multi-label classifiers.

Using ensemble techniques Sanden and Zhang \cite{sanden2011enhancing} were able to obtain better results in multi-label music genre classification than just using a single classifier. In this paper, our aim is to apply and evaluate various ensemble techniques on multi-label text classification.

The data set examined in this paper is a subset of the Aviation Safety Reporting System (ASRS) data set. The collection \emph{tmc2007} contains 28596 NASA aviation safety reports in free text form with 49060 discrete attributes corresponding to terms in the collection. Each document is represented as a term incidence vector. The safety reports are provided with 22 labels, each of them representing a problem type that appears during flights. For our purposes, subset containing 2000 randomly selected instances was used. In order to reduce the computational costs of experiments, we used a set of 500 features. The 500 features were selected in compliance with Tsoumakas and Vlahavas \cite{TsoumakasVlahavas2007kLablesets}. For each label the $\chi^{2}$ feature ranking method was used to obtain a ranking of all features for that label. The top 500 features were selected based on the their maximum rank over all labels \cite{TsoumakasVlahavas2007kLablesets}. Average cardinality in the collection is 2.2.

The rest of the paper is organized as follows. In the next section we provide a summary of related work. After that, we briefly describe multi-label classification algorithms, classifiers and ensemble techniques. In Section 4 we present the Experiment setup. In Section 5 we discuss the results and finally, Section 6 concludes our work.

\section{Related Work}
Ensemble methods combine results of multiple predictive models to achieve better performance than using any of the predictive models separately. Ensembles refer to a concrete finite set of alternative models and they seem to originate in bagging predictors \cite{breiman1996bagging}. Bagging predictors is a method that uses aggregated predictor. The tests show that bagging can provide substantial increase of accuracy. The ensemble method tends to produce better results with models showing high diversity among each other. 

Some work on ensemble techniques has been done by Shi, Kong, Yu and Wang \cite{kong2011ensemble}. They give a study of multi-label ensemble learning with focus on building a set of learners. Their proposed solution can efficiently improve the generalization ability of multi-label learning system and hence enhance the predictive performance of the classifier.

The work of Kubat, Sarinnapakorn and Dendamrongvit \cite{kubat2010induction}, deals with induction in multi-label text classification. They propose an induction technique of a set of subclassifiers that are applied on a same training set but use different features, and how to combine their outputs.

Sanden and Zhang \cite{sanden2011enhancing} propose a set of ensemble techniques to obtain better results as with individual multi-label classification algorithms. These techniques also help to overcome the drawbacks of individual classifiers. Their experimental study deals with music genre classification but can be beneficial for other domains as well. 


\section{Multi-label Classification Algorithms}
The task of multi-label classification is to produce output of $(d_{i}, L_{i})$ from a collection of possible labels $C = \{c_{1}, c_{2},...,c_{N}\}$ for each document in the test dataset of $D_{t} = \{d_{1}, d_{2},...,d_{m}\}$, given the training dataset $D_{r} =\{(d_{1},L_{1}),(d_{2},L_{2}),...,(d_{n},L_{n})\}$, where $L_{i} \subseteq C$. The approach performed to solve this task can be divided into two categories, \emph{problem transformation method} and \emph{algorithm adaptation method} \cite{MLCOverview}.

The \emph{problem transformation method} works by transforming the multi-label classification problem into one or multiple of single label classification problem. Hence, using the single label classifier as a base of multi-label classifier. The \emph{algorithm adaptation method} works by handling the multi-label classification problem directly, by extending the capability of certain classification algorithm.

We use five different multi-label classification algorithms as components for the ensemble techniques. The detail of the algorithms can be found in \cite{MLLChapter} and 	\cite{MLLSlides}.

\emph{RAkEL}\\
\emph{Random k-Labelset} (\emph{RAkEL}) randomly create $n$ different subset $C_{i} \subseteq C$ of label set $C$ with each having $k$ distinct labels. The classification model for each $C_{i}$ is built using \emph{Label Powerset} (\emph{LP}) method that treats each member $c_{ij}$ of powerset of $C_{i}$, $P(C_{i}) = \{\{\},\{c_{i1}\},\{c_{i2}\},...,\{c_{i1},c_{i2},..,c_{ik}\}\}$ as a single label, and use single label classifier to produce the model. The output from $n$ different models of \emph{LP} classifier are combined to get the final multi-label classification result.

\emph{CLR}\\
\emph{Calibrated Label Ranking} (\emph{CLR}) produces $q^{2}+q$ different pairs of distinct labels ($c_{i}, c_{j}$), where $q$ is the number of labels in $C$, with an additional of one virtual label $v$ that is used to differentiate the positive and negative labels in the final classification. A model is built for each pair using a single label classifier that only takes training data which contains $c_{i}$ or $c_{j}$ (but not both) as its label. The final classification result is produced by combining all models.

\emph{ML-kNN}\\
\emph{Multi-label k-Nearest Neighbour} (\emph{ML-kNN}) extends the idea of \emph{kNN} method to perform a multi-label classification. Given a test document $d$, we identify $N(d)$ as the $k$ nearest neighbours of $d$. The $q$-dimensional vector $\vec{C_{d}}$ is created where the $i$-th dimension of $\vec{C_{d}}$ represent the number of members in $N(d)$ having the $i$-th label. The final classification result is calculated using \emph{Maximum A Posteriori} (\emph{MAP}) principle, that estimate how likely for $d$ to have the $i$-th label given its $j$ ($j\le k$) nearest neighbours have the $i$-th label.

\emph{HOMER}\\
\emph{Hierarchy of Multilabel Classifiers} (\emph{HOMER}) works by constructing a hierarchy tree of labels, with each node considered to have a single label. At the lowest level of the hierarchy, the labels $C$ are divided into $n$ cluster using a \emph{balance clustering algorithm}, with labels in the same cluster have the same parent. Internal node of the tree is given a meta-label $\mu$ and the value is determined by a disjunction of the labels of its childrens $\mu = \bigvee\limits_{j}{}c_{j}$. A model is built for each group of labels that have the same parent, from lowest level to the root, using a multi-label classifier. The classification process starts from the root and recursively processing a test document $d$ until the lowest level to get the resulting labels.

\emph{IBLR}\\
\emph{Instance Based Logistic Regression} (\emph{IBRL}) combines the instance based learner algorithm with logistic regression method. The basic idea is to consider the labels of neighbouring instances or documents as additional features. This approach is to ensure that the interdependencies between class labels is taken into the classification. More detailed explanation of this algorithm can be found in \cite{IBLR}.


\subsection{Ensemble Techniques}
In this paper, we adapt the ensemble techniques presented in \cite{sanden2011enhancing} into our experiment. Basically, for a test document $d$, a multi-label classifier $K_{j}$ produces two kind of $N$-dimensional vectors, a score vector and a bipartition vector. The score vector $\vec{S^{j}} = \{s^{j}_{1}, s^{j}_{2},..,s^{j}_{N}\}$ contains probability or confidence values $s^{j}_{i}$ for $i$-th label assigned by a classifier $K_{j}$. The bipartition vector $\vec{B^{j}} = \{b^{j}_{1}, b^{j}_{2},..,b^{j}_{N}\}$ contains binary prediction values $b^{j}_{i}$ with value 1 if the classifier predict document $d$ can be assigned to $i$-th label and 0 otherwise. The ensemble techniques presented below are categorized based on the type of the output of the classifier.

\subsubsection{Bipartition-based Ensemble}
Bipartition-based ensemble takes bipartition vector $\vec{B^{j}}$ from each classification algorithms and combine them together to get the final multi-label classification. We denote the resulting bipartition vector as $\vec{B^{ens}} = \{b^{ens}_{1}, b^{ens}_{2},..,b^{ens}_{N}\}$. The operation to combine the vectors can use simple boolean operations or by simply calculating the number of occurrences of the positive classification for each label. 

\emph{Intersection Rule}\\
The \emph{Intersection Rule} use an AND boolean operation on each column $i$ of each vector $\vec{B^{j}}$, denoted as $b^{ens}_{i} = \bigwedge\limits_{j}{}b^{j}_{i}$. This rule represents the agreement by all classifiers.

\emph{Union Rule}\\
The \emph{Union Rule} use an OR boolean operation. In order to get the result, each column $i$ in vector $\vec{B^{j}}$ is combined as $b^{ens}_{i} = \bigvee\limits_{j}{}b^{j}_{i}$. A document will be assigned the $i$-th label if at least one of the classifier give value 1 for the label.

\emph{Majority Vote Rule}\\
The \emph{Majority Vote Rule} take the majority of the label assigned by the classifiers, and can be denoted as:
\[b^{ens}_{i} = \left\{
\begin{array}{cl}
1 & \textnormal{ if } A(1) \ge A(0) \\
0 & \textnormal{ otherwise}
\end{array}\right.\]
where $A(1)$ is the number of classifiers that give value 1 for $i$-th label and $A(0)$ the number of classifiers that give value 0.

\subsubsection{Score-based Ensemble}
Score-based ensemble works on score vector $\vec{S}$ of the classification algorithms. We denote the resulting score-based vector as $\vec{S^{ens}} = \{s^{ens}_{1}, s^{ens}_{2},..,s^{ens}_{N}\}$. The resulting classification is determined by using comparisons or by averaging the value for each label.

\emph{Minimum Rule}\\
The \emph{Minimum Rule} takes the lowest score assigned by classifiers for each $i$-th label. It is calculated as: 
\[s^{ens}_{i} = min_{j}(s^{j}_i)\]

\emph{Maximum Rule}\\
Contrary to the \emph{Minimum Rule}, the \emph{Maximum Rule} takes the highest score assigned by classifiers for each $i$-th label. It is calculated as:
\[s^{ens}_{i} = max_{j}(s^{j}_i)\]

\emph{Mean Rule}\\
The \emph{Mean Rule} takes an average of the value for $i$-th label from all classifiers. For each column $i$, the value is calculated as:
\[s^{ens}_{i} = \sum\limits_{j}{}s^{j}_i / M\]
where $M$ is the number of classifiers used.

\emph{Top-k Rule}\\
\emph{Top-k Rule} is proposed in \cite{sanden2011enhancing}, that takes an average of the $k$ largest values only. The value $k$ is a constant determined in advance. The value is calculated as:
\[s^{ens}_{i} = avg(topk_{j}(s^{j}_{i}))\]


\section{Experiment Setup}
To perform the evaluation, we used the Mulan \cite{mulan} open source
library. We implemented the ensemble techniques on top of the provided
interfaces and used the included evaluation framework to perform
10-fold cross-validation for all the individual multi-label learners
and the ensemble techniques.

The dataset used was obtained from the Mulan website
(http://mulan.sourceforge.net/datasets.html). The 28596 instances of
the Text Mining Challenge were randomly stripped down to about 2000
instances to make running the experiments feasible on our
equipment. Instead of the full data where every document is
represented by 49060 term incidence booleans, we used a stripped down
version processed by feature selection which uses only the 500 most
important terms. Again, this was done to make the execution of the
experiments viable on our machines.

The 5 constituent classifiers were set up using provided default or
customary settings. This means that RAkEL was using Label Powerset as
the internal multi-label learner which in turn used J48 decision trees
for single-label classification. CLR used SVM as the internal
classifier, which was trained using the SMO learner with a linear
kernel. ML-kNN and IBLR were initialized using the default
implementation of their constructors. For HOMER, we used Binary
Relevance as the internal classifier which in turn used SVMs for
binary classification. The number of clusters was set to 2 and the
balanced clustering method was used (these settings were taken from
the evaluations done in \cite{HOMER}).

\section{Results}

\begin{table*}[ht]
	\begin{center}
	\label{tab:bipartition-table}
		\begin{tabular}{rccccccc}
		\toprule
			 & HL & SA & Recall & Accu. & MicroP & MicroR & MicroF_1\\
			\midrule
			RAkEL & $0.0702$ & $0.2310$ & $0.5976$ & $0.4967$ & $0.6811$ & $0.5578$ & $0.6127$\\
			CLR & $0.0699$ & $0.2349$ & $0.6841$ & $\textbf{0.5322}$ & $0.6503$ & $0.6475$ & $\textbf{0.6485}$\\
			ML-kNN & $0.0762$ & $0.1697$ & $0.5041$ & $0.4335$ & $0.6666$ & $0.4720$ & $0.5518$\\
			HOMER & $0.0799$ & $0.2115$ & $0.6604$ & $0.5029$ & $0.5958$ & $0.6218$ & $0.6081$\\
			IBLR & $0.0752$ & $0.1772$ & $0.5209$ & $0.4429$ & $0.6679$ & $0.4890$ & $0.5642$\\
			Intersection & $0.0772$ & $0.1433$ & $0.3173$ & $0.3078$ & $\textbf{0.8354}$ & $0.2823$ & $0.4214$\\
			Union & $0.0929$ & $0.1513$ & $\textbf{0.8474}$ & $0.5202$ & $0.5221$ & $\textbf{0.8194}$ & $0.6374$\\
			Majority vote & $\textbf{0.0639}$ & $\textbf{0.2559}$ & $0.6153$ & $0.5300$ & $0.7300$ & $0.5709$ & $0.6402$\\
		\bottomrule
		\end{tabular}
                \caption{Experiment results for the bipartition-based ensemble techniques.}
	\end{center}
\end{table*}

\begin{table*}[ht]
	\begin{center}
	\label{tab:score-table}
		\begin{tabular}{rccccccc}
		\toprule
			 & AP & CO & OE & IE & ESS & RL & MicroAUC\\
			\midrule
			RAkEL & $0.7385$ & $5.9043$ & $0.2469$ & $0.5898$ & $5.9793$ & $0.1263$ & $0.8673$\\
			CLR & $0.8023$ & $2.8701$ & $0.2349$ & $\textbf{0.5067}$ & $2.2828$ & $0.0507$ & $0.9399$\\
			ML-kNN & $0.7204$ & $4.2654$ & $0.3126$ & $0.6371$ & $4.0329$ & $0.0930$ & $0.9048$\\
			HOMER & $0.6276$ & $8.8283$ & $0.3579$ & $0.6805$ & $10.7272$ & $0.2261$ & $0.7875$\\
			IBLR & $0.7317$ & $4.0045$ & $0.2972$ & $0.6222$ & $3.7347$ & $0.0855$ & $0.9098$\\
			Minimum & $0.6344$ & $9.3977$ & $0.2747$ & $0.6765$ & $11.5653$ & $0.2450$ & $0.7716$\\
			Maximum & $0.7535$ & $2.9771$ & $0.3484$ & $0.5734$ & $2.6181$ & $0.0585$ & $0.9328$\\
			Mean & $0.8039$ & $\textbf{2.8153}$ & $0.2314$ & $0.5102$ & $2.2100$ & $0.0496$ & $\textbf{0.9477}$\\
			Top\textsubscript{3} & $\textbf{0.8082}$ & $2.8228$ & $\textbf{0.2140}$ & $0.5082$ & $\textbf{2.2085}$ & $\textbf{0.0495}$ & $0.9475$\\
		\bottomrule
		\end{tabular}
                \caption{Experiment results for the score-based ensemble techniques.}
	\end{center}
\end{table*}


The results of our experiments can be seen in Tables 1 and 2 with the
best achieved values highlighted in bold-face.

First, we give an explanation of the measures and their abbreviations
used in the two tables. HL (Hamming Loss), SA (Subset Accuracy),
Recall (Example-based Recall), Accu.\ (Example-based Accuracy), MicroP
(Micro-averaged Precision), MicroR (Micro-averaged Recall),
MicroF\textsubscript{1} (Micro-averaged F\textsubscript{1}), AP
(Average Precision), CO (Coverage), OE (One Error) and RL (Ranking
Loss) are all evaluation measures described in \cite{MLLSlides}. IE
(Is Error) is the relative frequency of the predicted labelset being
different from the true labelset. ESS (Error Set Size) represents the
number of label pairs where an unrelevant label was ranked above a
relevant one and is thus basically isomorphic to the Ranking Loss
measure. MicroAUC is the micro-averaged area under the ROC curve.

Some expected statistics are conspicuously missing. Example-based
precision is not given since for some examples, the positive rate of
the classifier might be zero and precision is thus not
defined. Therefore, the example-based precision, which is meant to be
the average of such precision values, is not defined either. The same
goes for the example-based F\textsubscript{1} measure which is a
function of the precision and recall measures.

Also missing are all macro-averaged measures. This follows from the
fact that for some label, the statistic cannot be defined due to the
contingency tables being degenerate. Therefore, an average over
undefined values stays undefined. Micro-averaged measures, on the
other hand, are fine, as they average the contingency tables for all
the labels and then compute the statistics from the final contingency
table, which eliminates the probability of the contingency table being
degenerate.

\subsection*{Analyzing the Results}

Let us start with the bipartition-based classifiers whose results are
posted in Table~1. For all the first four example-based measures, CLR
seems to be the best individual classifier, which might lead us to
think that the ensemble techniques will fair worse as no measure would
make us prefer any other method. The micro-averaged measures however
reveal that some methods might be actually advantageous in some
situations (see RAkEL's micro-averaged precision, which is higher than
that of CLR). This paints a different picture than
\cite{sanden2011enhancing} where CLR was not the best performer and if
it excelled in something, it was precision. This goes to show that
different classifiers end up being more or less useful given the data
they are used on.

When we consider the ensemble techniques, performance tends to
increase in some measures and decrease in others. The majority vote
technique ends up being better in Hamming Loss and Subset Accuracy,
but loses to CLR in Accuracy and Micro-averaged
F\textsubscript{1}. This leads us to believe that bipartition-based
ensemble techniques do not offer a significant improvement in general
performance. However, one-sided measures like precision and recall can
be greatly improved by using the intersection and union techniques
which might be handy for specific applications.

Let us now turn to the results yielded by the score-based classifiers
on display in Table~2. The individual classifiers are clearly
dominated by CLR which offers the best performance for all the
evaluation metrics, confirming its appropriateness for the problem at
hand. In face of this one-sided result, we might not expect the
ensemble methods to provide much of an improvement. However, in all of
the metrics but IE, the mean and top\textsubscript{3} ensemble
techniques offer better performance than CLR alone. This corroborates
the results seen in \cite{sanden2011enhancing}, where the mean and
top\textsubscript{3} techniques consistently beat the individual
classifiers as well. Similarly to \cite{sanden2011enhancing},
top\textsubscript{3} seems to be the better of the two techniques.

\section{Conclusion}
We have seen that the ensemble techniques presented in
\cite{sanden2011enhancing} have universal applications and can be
easily used for text classification. We have seen that the
top\textsubscript{3} and mean ensemble techniques are the best
performers as in Sanden's and Zhang's research. In our situation, one
of the preexisting classifiers dominated the other ones in
performance, yet still the ensemble techniques benefited from
including all of them. Finally, we have also discovered that CLR seems
to be a very useful multi-label learner for text classification with a
small amount of labels.

This work could be continued by examining more sophisticated ways of
integrating the individual classifiers into an ensemble classifier. We
might also try adding different multi-label learners to the mix or try
creating ensemble classifiers using only some learners which perform
exceedingly well. Another direction might be to try and apply ensemble
techniques to another problem or field.


% The following two commands are all you need in the
% initial runs of your .tex file to
% produce the bibliography for the citations in your paper.
\bibliographystyle{abbrv}
\bibliography{paper,paper-Martin}
% You must have a proper ".bib" file
%  and remember to run:
% latex bibtex latex latex
% to resolve all references

\balancecolumns
% That's all folks!
\end{document}
