\section{Experiment Setup}
This would be the place to mention the implementation of the ensemble
techniques, the use of Mulan, the selection and format of the data. Estimated
1/4 of a page.

\section{Results}

\begin{table*}[ht]
	\begin{center}
	\label{tab:bipartition-table}
		\begin{tabular}{rccccccc}
		\toprule
			 & HL & SA & Recall & Accu. & MicroP & MicroR & MicroF_1\\
			\midrule
			RAkEL & $0.0702$ & $0.2310$ & $0.5976$ & $0.4967$ & $0.6811$ & $0.5578$ & $0.6127$\\
			CLR & $0.0699$ & $0.2349$ & $0.6841$ & $\textbf{0.5322}$ & $0.6503$ & $0.6475$ & $\textbf{0.6485}$\\
			ML-kNN & $0.0762$ & $0.1697$ & $0.5041$ & $0.4335$ & $0.6666$ & $0.4720$ & $0.5518$\\
			HOMER & $0.0799$ & $0.2115$ & $0.6604$ & $0.5029$ & $0.5958$ & $0.6218$ & $0.6081$\\
			IBLR & $0.0752$ & $0.1772$ & $0.5209$ & $0.4429$ & $0.6679$ & $0.4890$ & $0.5642$\\
			Intersection & $0.0772$ & $0.1433$ & $0.3173$ & $0.3078$ & $\textbf{0.8354}$ & $0.2823$ & $0.4214$\\
			Union & $0.0929$ & $0.1513$ & $\textbf{0.8474}$ & $0.5202$ & $0.5221$ & $\textbf{0.8194}$ & $0.6374$\\
			Majority vote & $\textbf{0.0639}$ & $\textbf{0.2559}$ & $0.6153$ & $0.5300$ & $0.7300$ & $0.5709$ & $0.6402$\\
		\bottomrule
		\end{tabular}
                \caption{Experiment results for the bipartition-based ensemble techniques.}
	\end{center}
\end{table*}

\begin{table*}[ht]
	\begin{center}
	\label{tab:score-table}
		\begin{tabular}{rccccccc}
		\toprule
			 & AP & CO & OE & IE & ESS & RL & MicroAUC\\
			\midrule
			RAkEL & $0.7385$ & $5.9043$ & $0.2469$ & $0.5898$ & $5.9793$ & $0.1263$ & $0.8673$\\
			CLR & $0.8023$ & $2.8701$ & $0.2349$ & $\textbf{0.5067}$ & $2.2828$ & $0.0507$ & $0.9399$\\
			ML-kNN & $0.7204$ & $4.2654$ & $0.3126$ & $0.6371$ & $4.0329$ & $0.0930$ & $0.9048$\\
			HOMER & $0.6276$ & $8.8283$ & $0.3579$ & $0.6805$ & $10.7272$ & $0.2261$ & $0.7875$\\
			IBLR & $0.7317$ & $4.0045$ & $0.2972$ & $0.6222$ & $3.7347$ & $0.0855$ & $0.9098$\\
			Minimum & $0.6344$ & $9.3977$ & $0.2747$ & $0.6765$ & $11.5653$ & $0.2450$ & $0.7716$\\
			Maximum & $0.7535$ & $2.9771$ & $0.3484$ & $0.5734$ & $2.6181$ & $0.0585$ & $0.9328$\\
			Mean & $0.8039$ & $\textbf{2.8153}$ & $0.2314$ & $0.5102$ & $2.2100$ & $0.0496$ & $\textbf{0.9477}$\\
			Top\textsubscript{3} & $\textbf{0.8082}$ & $2.8228$ & $\textbf{0.2140}$ & $0.5082$ & $\textbf{2.2085}$ & $\textbf{0.0495}$ & $0.9475$\\
		\bottomrule
		\end{tabular}
                \caption{Experiment results for the score-based ensemble techniques.}
	\end{center}
\end{table*}


The results of our experiments can be seen in Tables 1 and 2 with the
best achieved values highlighted in bold-face.

First, we give an explanation of the measures and their abbreviations
used in the two tables. HL (Hamming Loss), SA (Subset Accuracy),
Recall (Example-based Recall), Accu.\ (Example-based Accuracy), MicroP
(Micro-averaged Precision), MicroR (Micro-averaged Recall),
MicroF\textsubscript{1} (Micro-averaged F\textsubscript{1}), AP
(Average Precision), CO (Coverage), OE (One Error) and RL (Ranking
Loss) are all evaluation measures described in \cite{MLLSlides}. IE
(Is Error) is the relative frequency of the predicted labelset being
different from the true labelset. ESS (Error Set Size) represents the
number of label pairs where an unrelevant label was ranked above a
relevant one and is thus basically isomorphic to the Ranking Loss
measure. MicroAUC is the micro-averaged area under the ROC curve.

Some expected statistics are conspicuously missing. Example-based
precision is not given since for some examples, the positive rate of
the classifier might be zero and precision is thus not
defined. Therefore, the example-based precision, which is meant to be
the average of such precision values, is not defined either. The same
goes for the example-based F\textsubscript{1} measure which is a
function of the precision and recall measures.

Also missing are all macro-averaged measures. This follows from the
fact that for some label, the statistic cannot be defined due to the
contingency tables being degenerate. Therefore an average over
undefined values stays undefined. Micro-averaged measures on the other
hand are fine, as they average the contingency tables for all the
labels and then compute the statistics from the final contingency
table, which eliminates the probability of the contingency table being
degenerate.

\subsection*{Analyzing the Results}

Let us start with the bipartition-based classifiers whose results are
posted in Table~1. For all the first four example-based measures, CLR
seems to be the best individual classifier, which might lead us to
think that the ensemble techniques will fair worse as no measure would
make us prefer any other method. The micro-averaged measures however
reveal that some methods might be actually advantageous in some
situations (see RAkEL's micro-averaged precision, which is higher than
that of CLR). This paints a different picture than
\cite{sanden2011enhancing} where CLR was not the best performer and if
it excelled in something, it was precision. This goes to show that
different classifiers end up being more or less useful given the data
they are used on.

When we consider the ensemble techniques, performance tends to
increase in some measures and decrease in others. The majority vote
technique ends up being better in Hamming Loss and Subset Accuracy,
but loses to CLR in Accuracy and Micro-averaged
F\textsubscript{1}. This leads us to believe that bipartition-based
ensemble techniques do not offer a significant improvement of general
performance. However, one-sided measures like precision and recall can
be greatly improved by using the intersection and union techniques
which might be handy for specific applications.

Let us now turn to the results yielded by the score-based classifiers
on display on Table~2. The individual classifiers are clearly
dominated by CLR which offers the best performance for all the
evaluation metrics, confirming its appropriateness for the problem at
hand. In face of this one-sided result, we might not expect the
ensemble methods to provide much of an improvement. However, in all of
the metrics but IE, the mean and top\textsubscript{3} ensemble
techniques offer better performance than CLR alone. This corroborates
the results seen in \cite{sanden2011enhancing}, where the mean and
top\textsubscript{3} techniques consistently beat the individual
classifiers as well. Similarly to \cite{sanden2011enhancing},
top\textsubscript{3} seems to be the better of the two techniques.

\section{Conclusion}
A summary of what we observed in the results and the conclusions reached from
there. Estimated 1/2 of a page (including the references).
